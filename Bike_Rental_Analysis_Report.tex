% Options for packages loaded elsewhere
\PassOptionsToPackage{unicode}{hyperref}
\PassOptionsToPackage{hyphens}{url}
%
\documentclass[
]{article}
\usepackage{amsmath,amssymb}
\usepackage{iftex}
\ifPDFTeX
  \usepackage[T1]{fontenc}
  \usepackage[utf8]{inputenc}
  \usepackage{textcomp} % provide euro and other symbols
\else % if luatex or xetex
  \usepackage{unicode-math} % this also loads fontspec
  \defaultfontfeatures{Scale=MatchLowercase}
  \defaultfontfeatures[\rmfamily]{Ligatures=TeX,Scale=1}
\fi
\usepackage{lmodern}
\ifPDFTeX\else
  % xetex/luatex font selection
\fi
% Use upquote if available, for straight quotes in verbatim environments
\IfFileExists{upquote.sty}{\usepackage{upquote}}{}
\IfFileExists{microtype.sty}{% use microtype if available
  \usepackage[]{microtype}
  \UseMicrotypeSet[protrusion]{basicmath} % disable protrusion for tt fonts
}{}
\makeatletter
\@ifundefined{KOMAClassName}{% if non-KOMA class
  \IfFileExists{parskip.sty}{%
    \usepackage{parskip}
  }{% else
    \setlength{\parindent}{0pt}
    \setlength{\parskip}{6pt plus 2pt minus 1pt}}
}{% if KOMA class
  \KOMAoptions{parskip=half}}
\makeatother
\usepackage{xcolor}
\usepackage[margin=1in]{geometry}
\usepackage{graphicx}
\makeatletter
\def\maxwidth{\ifdim\Gin@nat@width>\linewidth\linewidth\else\Gin@nat@width\fi}
\def\maxheight{\ifdim\Gin@nat@height>\textheight\textheight\else\Gin@nat@height\fi}
\makeatother
% Scale images if necessary, so that they will not overflow the page
% margins by default, and it is still possible to overwrite the defaults
% using explicit options in \includegraphics[width, height, ...]{}
\setkeys{Gin}{width=\maxwidth,height=\maxheight,keepaspectratio}
% Set default figure placement to htbp
\makeatletter
\def\fps@figure{htbp}
\makeatother
\setlength{\emergencystretch}{3em} % prevent overfull lines
\providecommand{\tightlist}{%
  \setlength{\itemsep}{0pt}\setlength{\parskip}{0pt}}
\setcounter{secnumdepth}{-\maxdimen} % remove section numbering
\ifLuaTeX
  \usepackage{selnolig}  % disable illegal ligatures
\fi
\usepackage{bookmark}
\IfFileExists{xurl.sty}{\usepackage{xurl}}{} % add URL line breaks if available
\urlstyle{same}
\hypersetup{
  pdftitle={Draft},
  pdfauthor={Kasper Kuznik},
  hidelinks,
  pdfcreator={LaTeX via pandoc}}

\title{Draft}
\author{Kasper Kuznik}
\date{2024-11-13}

\begin{document}
\maketitle

\section{1. Introduction}\label{introduction}

The rapid urbanisation of recent decades has seen cities face challenges
such as traffic congestion, air pollution, and limited parking spaces.
In the face of these issues, rental bike systems have gained significant
popularity, offering a cheap and eco-friendly alternative to more
traditional commuting methods. Rental bikes provide a practical solution
for short-distance travel, be it for students going to school, office
workers heading to the workplace and even tourists exploring a city. In
addition to promoting an active lifestyle, they eliminate the hassle of
bike ownership, such as storage and maintenance. The growing demand of
rental bike companies is reflected in the global market value of bike
rentals, projected to reach a value of \$11.3 billion in 2031, marking a
538\% increase from the 2021 market value of \$2.1 billion (Allied
Market Research, 2021). The biggest challenge rental bike companies face
is ensuring constant availability of bikes, as demand can fluctuate due
to numerous reasons. Bikes must be readily accessible to users at all
times, making it essential for company leaders to be aware of peak
hours, the impact of weather conditions and high-demand areas. In this
project, we will use various machine learning models to predict the
number of rented bikes required to meet customer demand under different
circumstances, trying to figure out which of them performs best. The
goal is to assist a fictional bike rental company operating in Seoul,
South Korea, in planning the optimal size of their bike fleet.

\section{Dataset „Seoul Bike Sharing
Demand``}\label{dataset-seoul-bike-sharing-demand}

To conduct our analysis, we will use a dataset which contains the count
of public bicycles rented per hour in the Seoul Bike Sharing System with
corresponding weather data and holiday information, collected between
2017 and 2018 (UCI Machine Learning Repository, 2020). We will enrich
the dataset by adding columns that capture different times-related
factors, such as whether a given day is a working day or a weekend.
Additionally, we will categorise hours into specific periods to better
capture daily trends in bike demand.

\begin{enumerate}
\def\labelenumi{\arabic{enumi}.}
\setcounter{enumi}{1}
\item
  Data Transformation
\item
  Exploratory data analysis
\item
  Modeling
\end{enumerate}

\begin{enumerate}
\def\labelenumi{\alph{enumi})}
\item
  Linear Model i) Model implementation ii) Result interpretation ii)
  Cross Validation*
\item
  Generalised Linear Model with family set to Poisson i) Model
  implementation ii) Result interpretation ii) Cross Validation*
\item
  Generalised Linear Model with family set to Binomial i) Model
  implementation ii) Result interpretation ii) Cross Validation*
\item
  Generalised Additive Model i) Model implementation ii) Result
  interpretation ii) Cross Validation*
\item
  Support Vector Machine (SVM) i) Model implementation ii) Result
  interpretation ii) Cross Validation*
\item
  Neural Network i) Model implementation ii) Result interpretation ii)
  Cross Validation*
\end{enumerate}

\begin{enumerate}
\def\labelenumi{\arabic{enumi}.}
\setcounter{enumi}{4}
\item
  Usage of AI
\item
  Conclusion
\end{enumerate}

\end{document}
