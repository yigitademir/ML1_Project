% Options for packages loaded elsewhere
\PassOptionsToPackage{unicode}{hyperref}
\PassOptionsToPackage{hyphens}{url}
%
\documentclass[
]{article}
\usepackage{amsmath,amssymb}
\usepackage{iftex}
\ifPDFTeX
  \usepackage[T1]{fontenc}
  \usepackage[utf8]{inputenc}
  \usepackage{textcomp} % provide euro and other symbols
\else % if luatex or xetex
  \usepackage{unicode-math} % this also loads fontspec
  \defaultfontfeatures{Scale=MatchLowercase}
  \defaultfontfeatures[\rmfamily]{Ligatures=TeX,Scale=1}
\fi
\usepackage{lmodern}
\ifPDFTeX\else
  % xetex/luatex font selection
\fi
% Use upquote if available, for straight quotes in verbatim environments
\IfFileExists{upquote.sty}{\usepackage{upquote}}{}
\IfFileExists{microtype.sty}{% use microtype if available
  \usepackage[]{microtype}
  \UseMicrotypeSet[protrusion]{basicmath} % disable protrusion for tt fonts
}{}
\makeatletter
\@ifundefined{KOMAClassName}{% if non-KOMA class
  \IfFileExists{parskip.sty}{%
    \usepackage{parskip}
  }{% else
    \setlength{\parindent}{0pt}
    \setlength{\parskip}{6pt plus 2pt minus 1pt}}
}{% if KOMA class
  \KOMAoptions{parskip=half}}
\makeatother
\usepackage{xcolor}
\usepackage[margin=1in]{geometry}
\usepackage{graphicx}
\makeatletter
\def\maxwidth{\ifdim\Gin@nat@width>\linewidth\linewidth\else\Gin@nat@width\fi}
\def\maxheight{\ifdim\Gin@nat@height>\textheight\textheight\else\Gin@nat@height\fi}
\makeatother
% Scale images if necessary, so that they will not overflow the page
% margins by default, and it is still possible to overwrite the defaults
% using explicit options in \includegraphics[width, height, ...]{}
\setkeys{Gin}{width=\maxwidth,height=\maxheight,keepaspectratio}
% Set default figure placement to htbp
\makeatletter
\def\fps@figure{htbp}
\makeatother
\setlength{\emergencystretch}{3em} % prevent overfull lines
\providecommand{\tightlist}{%
  \setlength{\itemsep}{0pt}\setlength{\parskip}{0pt}}
\setcounter{secnumdepth}{-\maxdimen} % remove section numbering
\ifLuaTeX
  \usepackage{selnolig}  % disable illegal ligatures
\fi
\usepackage{bookmark}
\IfFileExists{xurl.sty}{\usepackage{xurl}}{} % add URL line breaks if available
\urlstyle{same}
\hypersetup{
  pdftitle={Get on your bike},
  pdfauthor={Kuznik et al.},
  hidelinks,
  pdfcreator={LaTeX via pandoc}}

\title{Get on your bike}
\author{Kuznik et al.}
\date{2024-11-13}

\begin{document}
\maketitle

\section{1. Introduction}\label{introduction}

The rapid urbanisation of recent decades has seen cities face challenges
such as traffic congestion, air pollution, and limited parking spaces.
In the face of these issues, rental bike systems have gained significant
popularity, offering a cheap and eco-friendly alternative to more
traditional commuting methods. Rental bikes provide a practical solution
for short-distance travel, be it for students going to school, office
workers heading to the workplace and even tourists exploring a city. In
addition to promoting an active lifestyle, they eliminate the hassle of
bike ownership, such as storage and maintenance. The growing demand of
rental bike companies is reflected in the global market value of bike
rentals, projected to reach a value of \$11.3 billion in 2031, marking a
538\% increase from the 2021 market value of \$2.1 billion (Allied
Market Research, 2021). The biggest challenge rental bike companies face
is ensuring constant availability of bikes, as demand can fluctuate due
to numerous reasons. Bikes must be readily accessible to users at all
times, making it essential for company leaders to be aware of peak
hours, the impact of weather conditions and high-demand areas. In this
project, we will use various machine learning models to predict the
number of rented bikes required to meet customer demand under different
circumstances, trying to figure out which of them performs best. The
goal is to assist a fictional bike rental company operating in Seoul,
South Korea, in planning the optimal size of their bike fleet.

\section{Dataset}\label{dataset}

To conduct our analysis, we will use a dataset called „Seoul Bike
Sharing Demand`` which contains the count of public bicycles rented per
hour in the Seoul Bike Sharing System with corresponding weather data
and holiday information, collected between 2017 and 2018 (UCI Machine
Learning Repository, 2020). We will enrich the dataset by adding columns
that capture different times-related factors, such as whether a given
day is a working day or a weekend. Additionally, we will categorise
hours into specific periods to better capture daily trends in bike
demand.

\section{Structure}\label{structure}

We will begin by examining the dataset and outlining the specific
changes we made for the purpose of our analysis. Next, we will perform
exploratory data analysis, focussing on correlations between variables
to address potential multicollinearity issues. In the following
chapters, we will fit different machine learning models to the data and
use their results to predict the number of rental bikes needed in Seoul.
The sequence of the models will be as follows: a linear model, a
generalised linear model with a Poisson family, another with a Binomial
family, a generalised additive model, a neural network and a support
vector machine. Finally, we will compare these models to identify the
best solution for our bike rental company.

\section{2. Data Transformation}\label{data-transformation}

The raw dataset was unsuitable for analysis and therefore required
transformation. Seeing as the column names were unnecessarily complex
and inconsistent, the first step was renaming them for simplicity.
``Temperature.캜.'' was changed to ``Temperature'', ``Humidity\ldots{}''
to ``Humidity'', ``Visibility..10m.'' to ``Visibility'' and so forth. In
the categorical variable ``Functioning.Day'', which indicated whether a
given day was functioning for bike rentals, non-functioning days were
removed as the number of rented bikes was obviously zero on those days.
The column was subsequently dropped since its remaining values were
identical, making it redundant. Next, time-related features were added
to capture their potential impact on rented bikes. ``Month'' and
``Weekday'' columns were created to indicate the month and day of the
week, followed by an ``Is.Weekend'' column to classify days as working
days or weekends. To catch meaningful variations within a single day,
the ``Hours'' column was grouped into segments such as morning rush
hour, evening rush hour and late night. These segments were stored in a
new variable ``TimeOfDay''. Finally, all categorical variables were
converted to factor data types, completing the preparation of the
dataset for analysis.

\begin{enumerate}
\def\labelenumi{\arabic{enumi}.}
\setcounter{enumi}{2}
\item
  Exploratory data analysis
\item
  Modeling
\end{enumerate}

\begin{enumerate}
\def\labelenumi{\alph{enumi})}
\item
  Linear Model i) Model implementation ii) Result interpretation ii)
  Cross Validation*
\item
  Generalised Linear Model with family set to Poisson i) Model
  implementation ii) Result interpretation ii) Cross Validation*
\item
  Generalised Linear Model with family set to Binomial i) Model
  implementation ii) Result interpretation ii) Cross Validation*
\item
  Generalised Additive Model i) Model implementation ii) Result
  interpretation ii) Cross Validation*
\item
  Support Vector Machine (SVM) i) Model implementation ii) Result
  interpretation ii) Cross Validation*
\item
  Neural Network i) Model implementation ii) Result interpretation ii)
  Cross Validation*
\end{enumerate}

\begin{enumerate}
\def\labelenumi{\arabic{enumi}.}
\setcounter{enumi}{4}
\item
  Usage of AI
\item
  Conclusion
\end{enumerate}

\end{document}
